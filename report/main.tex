\documentclass[
a4paper,
11pt,
titlepage,
]{article}

%%%%%%%%%%%%%%%%%%%%%%%%%%%%%%%%%%%%%%%%%%%%%%%%%%%%%%%%%%%%%%%%%%%%%%
% GEOMETRY
%%%%%%%%%%%%%%%%%%%%%%%%%%%%%%%%%%%%%%%%%%%%%%%%%%%%%%%%%%%%%%%%%%%%%%
\usepackage[%
paper=a4paper,%
margin=2.5cm,%
]{geometry}


%%%%%%%%%%%%%%%%%%%%%%%%%%%%%%%%%%%%%%%%%%%%%%%%%%%%%%%%%%%%%%%%%%%%%%
% PACKAGES INCLUSION
%%%%%%%%%%%%%%%%%%%%%%%%%%%%%%%%%%%%%%%%%%%%%%%%%%%%%%%%%%%%%%%%%%%%%%


%%%%%%%%%%%%%%%%%%%%%%%%%%%%%%%%%%%%%%%%%%%%%%%%%%
% GENERIC PACKAGES
%%%%%%%%%%%%%%%%%%%%%%%%%%%%%%%%%%%%%%%%%%%%%%%%%%
\usepackage[utf8]{inputenc}					% Source code encoding (\'{e} --> é)
\usepackage[T1]{fontenc}					% Support accents in the output file
\usepackage[french]{babel}					% Support for french typography
\usepackage{setspace}						% Control vertical spacing between lines
\onehalfspacing%						% 1.5 spacing
\usepackage{graphicx}						% Graphics inclusion.
\graphicspath{{illustrations/}}				        % Path for images
\usepackage{pdfpages}						% PDF pages inclusion
\usepackage{ragged2e}						% Text alignement
\usepackage{fancybox}						% \shadowbox
\usepackage{xcolor}						% Colors
\usepackage{array}						% Tables
\usepackage{amsmath}						% Maths symbols
\usepackage{amssymb}						% Maths symbols
\usepackage[subrefformat=parens
]{subcaption}							% Subfigures -- incompatible with the subfigure package
\usepackage[margin=10pt,
font=footnotesize,
labelfont=bf,
labelsep=endash
]{caption}							% Vastly improves the standard formatting of captions
\usepackage{lipsum}						% Lorem ipsum
\usepackage[autostyle,
english=british,
french=quotes]{csquotes}					% Foreign language quotes
\usepackage{multicol}						% Multicolumn environments
\usepackage{listings}                 				% Source code displaying
% \usepackage{minted}						% Code syntax coloration


%%%%%%%%%%%%%%%%%%%%%%%%%%%%%%%%%%%%%%%%%%%%%%%%%%
% Appendices
%%%%%%%%%%%%%%%%%%%%%%%%%%%%%%%%%%%%%%%%%%%%%%%%%%
\usepackage[toc]{appendix}                                      % Appendices
\renewcommand{\appendixtocname}{Annexes}


%%%%%%%%%%%%%%%%%%%%%%%%%%%%%%%%%%%%%%%%%%%%%%%%%%
% REFERENCES / LINKS
%%%%%%%%%%%%%%%%%%%%%%%%%%%%%%%%%%%%%%%%%%%%%%%%%%
\usepackage[colorlinks=false,
linkbordercolor={1 0 0},
urlbordercolor={0 1 0},
citebordercolor={0 0 1},
]{hyperref} 							% Hyper-links management
\usepackage{url}						% Typesets URLs sensibly - with tt font, clickable in PDFs, and not breaking across lines


%%%%%%%%%%%%%%%%%%%%%%%%%%%%%%%%%%%%%%%%%%%%%%%%%%
% DRAWING
%%%%%%%%%%%%%%%%%%%%%%%%%%%%%%%%%%%%%%%%%%%%%%%%%%
\usepackage{tikz}						% Draw nice things
\usepackage{mathtools}						% especially the bloch sphere
\usetikzlibrary{arrows.meta}					% that needs more packages.
\usepackage{pgfplots}
\pgfplotsset{compat=1.13}
\usepgfplotslibrary{dateplot, statistics}
%\usepackage{filecontents}


%%%%%%%%%%%%%%%%%%%%%%%%%%%%%%%%%%%%%%%%%%%%%%%%%%%
%% GLOSSARY
%%%%%%%%%%%%%%%%%%%%%%%%%%%%%%%%%%%%%%%%%%%%%%%%%%%
%\usepackage[acronym,toc,style=index]{glossaries}
%\makeglossaries

%%%%%%%%%%%%%%%%%%%%%%%%%%%%%%%%%%%%%%%%%%%%%%%%%%
% NOMENCLATURE
%%%%%%%%%%%%%%%%%%%%%%%%%%%%%%%%%%%%%%%%%%%%%%%%%%
\usepackage[noprefix]{nomencl}
\makenomenclature%
\renewcommand{\nomname}{Liste des symboles}

\renewcommand\nomgroup[1]{
\item[\bfseries
\ifstrequal{#1}{G}{Lettres grecques}{
  \ifstrequal{#1}{L}{Lettres latines}{}}
]}

\newcommand{\nomunit}[1]{
\renewcommand{\nomentryend}{\hspace*{\fill}#1}}



%%%%%%%%%%%%%%%%%%%%%%%%%%%%%%%%%%%%%%%%%%%%%%%%%%
% BIBLIOGRAPHY
%%%%%%%%%%%%%%%%%%%%%%%%%%%%%%%%%%%%%%%%%%%%%%%%%%
\usepackage[backend=biber,
hyperref=true,
url=false,
isbn=true,
backref=true,
style=numeric,
sorting=none,
sortcites=true
]{biblatex}         % Bibliography management
\addbibresource{waves.bib}

%%%%%%%%%%%%%%%%%%%%%%%%%%%%%%%%%%%%%%%%%%%%%%%%%%
% FANCY COMMANDS
%%%%%%%%%%%%%%%%%%%%%%%%%%%%%%%%%%%%%%%%%%%%%%%%%%
\usepackage{fancyhdr}						% Allow fancy stuff in the page header
\pagestyle{fancy}
\fancypagestyle{plain}

% Header
\renewcommand{\headrulewidth}{1pt}
% Footer
\fancyfoot[L]{}
\fancyfoot[R]{}
\fancyfoot[C]{\thepage/\pageref{myLastPage}}

\fancypagestyle{nofooter}{%
  \fancyfoot{}%
}

%%%%%%%%%%%%%%%%%%%%%%%%%%%%%%%%%%%%%%%%%%%%%%%%%%
% DEFINE SOME USEFUL MATH COMMANDS
%%%%%%%%%%%%%%%%%%%%%%%%%%%%%%%%%%%%%%%%%%%%%%%%%%
\newcommand{\uuline}[1]{\underline{\underline{#1}}}
\renewcommand{\vec}[1]{\mathbf{\underline{#1}}}
\newcommand{\tens}[1]{\mathbf{\uuline{#1}}}

%%%%%%%%%%%%%%%%%%%%%%%%%%%%%%%%%%%%%%%%%%%%%%%%%%
% LAST WORDS BEFORE \BEGIN{DOCUMENT}
%%%%%%%%%%%%%%%%%%%%%%%%%%%%%%%%%%%%%%%%%%%%%%%%%%
\everymath{\displaystyle}


%%%%%%%%%%%%%%%%%%%%%%%%%%%%%%%%%%%%%%%%%%%%%%%%%%%%%%%%%%%%%%%%%%%%%%
% DOCUMENT START
%%%%%%%%%%%%%%%%%%%%%%%%%%%%%%%%%%%%%%%%%%%%%%%%%%%%%%%%%%%%%%%%%%%%%%
\begin{document}

%%%%%%%%%%%%%%%%%%%%%%%%%%%%%%%%%%%%%%%%%%%%%%%%%%
% TITLE PAGE
%%%%%%%%%%%%%%%%%%%%%%%%%%%%%%%%%%%%%%%%%%%%%%%%%%
\pagenumbering{gobble} % Do not number the title page.
\thispagestyle{empty}
\begin{titlepage}
	
  \begin{multicols}{2}
    \noindent ENSEIRB-MATMECA\\
    1 Avenue du Dr Albert Schweitzer\\
    33400 Talence\\
    \url{https://enseirb-matmeca.bordeaux-inp.fr/fr}
    \begin{flushright}
      \includegraphics[height=3cm]{logo_enseirb_matmeca.png}
    \end{flushright}
  \end{multicols}
  
  \begin{center}
    \rule{\linewidth}{0.5mm}\\[0.5cm]
    {\Large \textbf{Rapport de TER}}\\[0.3cm]
    \rule{\linewidth}{0.5mm}\\[1.5cm]
    
    \textbf{Année 2020 - 2021}\\[1.5cm]	
    
    \textbf{Modélisation de la propagation des vagues en milieu littoral}\\[3cm]
    
    \begin{tabular}{rl}
      {\it Auteurs} & Robin {\sc Colombier}\\
                    & Théo {\sc Guichard}\\
                    & Geoffrey {\sc Lebaud}\\
                    & Rémi {\sc Pégouret}\\
                    & Gabriel {\sc Suau}\\
                    & Lucas {\sc Trautmann}\\
                    &\\
      {\it Tuteurs} & Natalie {\sc Bonneton}\\
                    & Rodolphe {\sc Turpault}\\
    \end{tabular}\\[2cm]
  \end{center}
\end{titlepage}

\nocite{*}

\clearpage

\pagenumbering{gobble} % Do not number the contents page.
\thispagestyle{empty}  % No fancy header/foter.

\tableofcontents

\newpage

\pagenumbering{arabic} % We want the numerotation now.
\fancypagestyle{plain} % Now we want the fancy things.

\section*{Résumé}
\addcontentsline{toc}{section}{Résumé}

\section*{Abstract}
\addcontentsline{toc}{section}{Abstract}
\newpage

\printnomenclature[1cm]
\addcontentsline{toc}{section}{Liste des symboles}
\newpage

\section{Introduction}

\subsection{Propagation des vagues en milieu littoral}

\subsection{Dérivation des équations de Saint-Venant}

Dans cette section, nous détaillons la dérivation des équations de Saint-Venant 2D à partir des équations de Navier-Stokes. Tout d'abord, nous allons faire une première simplification en considérant l'eau comme un fluide parfait. L'écoulement est considéré incompressible, et est donc régi par les équations d'Euler incompressibles :
\begin{equation}
  \label{eq:eulerincomp}
  \rho\Big(\frac{\partial \vec{u}}{\partial t} + \big(\vec{u}\cdot\vec{\nabla}\big)\vec{u}\Big) = -\vec{\nabla}p + \rho\vec{g}
\end{equation}
\nomenclature[L]{$p$}{Pression \nomunit{Pa}}
\nomenclature[L]{$g$}{Accélération de la pesanteur \nomunit{$m.s^{-2}$}}
\nomenclature[L]{$\vec{u}$}{Vitesse du fluide \nomunit{$m.s^{-1}$}}
\nomenclature[G]{$\rho$}{Masse volumique \nomunit{$kg.m^{-3}$}}

Il y a trois équations pour quatre inconnues (la pression et les trois composantes de vitesse). Il faut fermer ce système d'équations en y ajoutant l'équation de conservation de la masse, qui s'écrit pour un écoulement incompressible :
\begin{equation}
  \label{eq:masscons}
  \vec{\nabla}\cdot\vec{u} = 0\\	
\end{equation}

On se place dans un repère cartésien $R(0,\vec{e_x},\vec{e_y},\vec{e_z})$ et on suppose que les variations verticales de vitesse sont faibles.
\subsection{Forme conservative des équations et hyperbolicité}

\section{Résolution numérique par volumes finis}

\section{Expériences sur le littoral Aquitain}

\newpage
\label{myLastPage}

\newpage

\printbibliography
\addcontentsline{toc}{section}{Références}

\begin{appendices}

  \section{Dérivation des équations de Saint-Venant}
  \label{sec:deriv-des-equat}

  \section{BlaBlaBla}
  \label{sec:blablabla}


\end{appendices}
\end{document}
